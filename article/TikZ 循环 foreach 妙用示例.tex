\documentclass[border=15pt]{standalone}
\usepackage{tikz}
\usetikzlibrary{calc}
\usepackage{ifthen}
\begin{document}
	\newcounter{maincounter}
	\newcounter{secondcounter}
	\begin{tikzpicture}[line width = 1.2pt]
		\pgfmathsetmacro{\h}{1.4}   %两点的水平间距
		\pgfmathsetmacro{\v}{1.4}   %两点的垂直间距
		% %------------------------------------------------------
		\foreach \r/\c in {0/0,2/0,4/0,6/0,0/2,2/2,4/2,6/2}
		{	
			\stepcounter{maincounter} %增加计数器的值
			\setcounter{secondcounter}{0}%设置计数器的值
			%定义原点坐标
			\coordinate  (P0) at (\r,\c);	
			
			\path (P0)  node[circle,fill=red,inner sep=2pt](a){$a$};
			\path (a)+(\h,0 )  node[circle,fill=red,inner sep=2pt](b){$b$};
			\path (b)+(0,\v )  node[circle,fill=red,inner sep=2pt](c){$c$};
			\path (a)+(0,\v )  node[circle,fill=red,inner sep=2pt](d){$d$};
			\path ($ (a)!0.5! (b) $)+(0,-0.25)   node (T){ $\left( \themaincounter \right) $};
			%画虚线
			\foreach \from/\to in {a/b,b/c,c/d,d/a,a/c,b/d}
			{
				\stepcounter{secondcounter} %增加计数器的值
				\ifthenelse{\value{maincounter} < 8}{
					\ifthenelse{\value{secondcounter} < \value{maincounter}}
					{\draw[blue,very thick] (\from)--(\to) ; }	{\draw[dashed,thin] (\from)--(\to) ;}
				}{\draw[dashed,thin] (\from)--(\to) ;}
				%------------------------------
			}
			%------------------------------
			\ifthenelse{\value{maincounter} = 8}{\draw[green,very thick] (a)--(b)--(c)--(a) ;  }{}			
		}
	\end{tikzpicture}
\end{document}