\documentclass{ctexart}
\usepackage{makecell}
\usepackage[top=1mm,bottom=1mm,left=1cm,right=1mm]{geometry}
\usepackage{array}
\usepackage{tabu}
\usepackage{amsmath}
% tabular 务必放在浮动环境 table 中否则会影响后文字体大小。
\usepackage{etoolbox}
\BeforeBeginEnvironment{tabular}{\bfseries\zihao{-6}}
\makeatletter
% version 1 use "
\newcommand{\thickhline}{%
	\noalign {\ifnum 0=`}\fi \hrule height 1pt
	\futurelet \reserved@a \@xhline
}
\newcolumntype{"}{@{\hskip\tabcolsep\vrule width 1pt\hskip\tabcolsep}}

\makeatother

\begin{document}
	\section{use table enviroment}
	\begin{table}[!h]

	\begin{tabular}{|c|c|c|}
		\hline
		\multicolumn{3}{|c|}{部分倒装与全部倒装的对比} \\
		\hline
		& 部分倒装 & 全部倒装 \\
		\hline
		概念 & \makecell*[c]{只把助动词、be动词、情态动词置于主语前;\\ 
			句首是程度副词} & \makecell*[c]{把“全部谓语动词”置于“主语”之前;\\ 
			句首是状语} \\
		\hline
		用法/条件 & \makecell*[l]{\hspace{2em}1. 否定词位于句首;\\ 
			\hspace{2em}2. “only + 状语”位于句首; \\ 
			\hspace{2em}3. such,so,well,often,many a time等 \\ 
			\hspace{2em}程度、频率副词于句首} & \makecell*[l]{\hspace{1em}1.主语不是代词;比如he, she, it ... \\ 
			\hspace{1em}2. 谓语动词不是及物动词! } \\
		\hline
		目的 & 为了突出句首的副词/状语 & 为了突出句尾的主语(除表语提前)  \\
		\hline
		区别 & 只看句首的副词/状语,对主语、谓语没要求 & 对主语、谓语有要求!(除表语提前) \\
		\hline
	\end{tabular}
		
\end{table}
normal text\\
正常文本
	\vspace{2ex}
	\hrule
	\section{use macro}
	\vspace{2ex}
	
	\begin{tabular}{"c"c"c"}
		\thickhline
		\multicolumn{3}{c}{部分倒装与全部倒装的对比} \\
		\thickhline
		& 部分倒装 & 全部倒装 \\
		\thickhline
		概念 & \makecell*[c]{只把助动词、be动词、情态动词置于主语前;\\ 
			句首是程度副词} & \makecell*[c]{把“全部谓语动词”置于“主语”之前;\\ 
			句首是状语} \\
		\thickhline
		用法/条件 & \makecell*[l]{\hspace{2em}1. 否定词位于句首;\\ 
			\hspace{2em}2. “only + 状语”位于句首; \\ 
			\hspace{2em}3. such,so,well,often,many a time等 \\ 
			\hspace{2em}程度、频率副词于句首} & \makecell*[l]{\hspace{1em}1.主语不是代词;比如he, she, it ... \\ 
			\hspace{1em}2. 谓语动词不是及物动词! } \\
		\thickhline
		目的 & 为了突出句首的副词/状语 & 为了突出句尾的主语(除表语提前)  \\
		\thickhline
		区别 & 只看句首的副词/状语,对主语、谓语没要求 & 对主语、谓语有要求!(除表语提前) \\
		\thickhline
	\end{tabular}\\
非正常文本
	
	\vspace{2ex}
	\hrule
	\vspace{2ex}
	\section{use tabu package}
	\begin{tabu}{|[2pt]c|c|c|}
		\tabucline[2pt]{-}
		\multicolumn{3}{|[2pt]c|[2pt]}{部分倒装与全部倒装的对比} \\
		\tabucline[2pt]{-}
		& 部分倒装 & 全部倒装 \\
		\tabucline[2pt]{-}
		概念 & \makecell*[c]{只把助动词、be动词、情态动词置于主语前;\\ 
			句首是程度副词} & \makecell*[c]{把“全部谓语动词”置于“主语”之前;\\ 
			句首是状语} \\
		\tabucline[2pt]{-}
		用法/条件 & \makecell*[l]{\hspace{2em}1. 否定词位于句首;\\ 
			\hspace{2em}2. “only + 状语”位于句首; \\ 
			\hspace{2em}3. such,so,well,often,many a time等 \\ 
			\hspace{2em}程度、频率副词于句首} & \makecell*[l]{\hspace{1em}1.主语不是代词;比如he, she, it ... \\ 
			\hspace{1em}2. 谓语动词不是及物动词! } \\
		\tabucline[2pt]{-}
		目的 & 为了突出句首的副词/状语 & 为了突出句尾的主语(除表语提前)  \\
		\tabucline[2pt]{-}
		区别 & 只看句首的副词/状语,对主语、谓语没要求 & 对主语、谓语有要求!(除表语提前) \\
		\tabucline[2pt]{-}
	\end{tabu}
	
	\section{color table}
	\subsection{use tabular enviroment}
	\begin{tabular}{lccc}
		\hline
		状态変化 & $Q^{in}$ & $\varDelta U$ & $W^{out}$ \\
		\hline \hline
		定量変化 & $nC_V\varDelta T$ & $nC_V\varDelta T$ & 0 \\
		定圧変化 & $nC_P\varDelta T$ & $nC_V\varDelta T$ & $nR\varDelta T$ \\
		等温変化 & $nRT\log\frac{V_1}{V_0}$ & 0 & $nRT\log\frac{V_1}{V_0}$ \\
		断熱変化変 & 0 & $nC_V\varDelta T$ & $-nC_V\varDelta T$ \\
		\hline
	\end{tabular}
\end{document}