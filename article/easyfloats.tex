\documentclass{ctexart}

% 引入easyfloats浮动体排版宏包
\usepackage[longtable]{easyfloats}
%% 设置easyfloats浮动体宏包
\objectset[table]{env=tabular, placement=htb}
\objectset[figure]{placement=htb}

\usepackage{siunitx}
\newcommand\pminfty{\multicolumn1r{$\pm\infty$}}

% 超链接宏包
\usepackage{hyperref}

% 最小工作示例宏包(插图)
\usepackage{mwe}

\title{使用\texttt{easyfloats}宏包排版浮动体}
\author{耿楠 \thanks{西北农林科技大学信息工程学院}}
\date{\today}

\begin{document}
	\maketitle
	
	\section{说明}
	由于\verb|easyfloats|是一个新的宏包,如果在你的发行版中还未包含该
	宏包,则可以在
	\href{https://gitlab.com/erzo/latex-easyfloats}{https://gitlab.com/erzo/latex-easyfloats}
	网站下载该宏包,并置于当前工作目录中以使用该宏包。
	\section{单个浮动体}
	\subsection{表格}
	可以使用\verb|tableobject|环境排版表格浮动体,通
	过\verb|<key>=<value>|设置排版参数,如表\ref{tab:test01}所示。
	\begin{tableobject}{caption=这是表格标题,
			label=tab:test01,
			placement=htb,    
			env=tabular}{cl}
		\toprule
		转义码 & 含义          \\
		\midrule
		0       & Escape Character \\
		1       & Begin Group      \\
		2       & End Group        \\
		\vdots  & \quad \vdots     \\
		\bottomrule
	\end{tableobject}
	
	\subsection{插图}
	可以使用\verb|\includegraphicobject|命令排版表格浮动体,通
	过\verb|<key>=<value>|设置排版参数,如图\ref{fig:example}所
	示。
	\includegraphicobject[%
	label   = fig:example,
	caption = 示例插图,
	details = 致谢\href{https://ctan.org}{www.ctan.org}.,
	graphic width = .5\linewidth,
	]{example-image}
	
	\section{子浮动体}
	可以使用\verb|subobject|环境实现子浮动体排版。
	\subsection{子表格}
	可以使用\verb|tableobject|环境结合\verb|subobject|环境实现子表格浮动体排版,通
	过\verb|<key>=<value>|设置排版参数,如图\ref{tab:tabs}所示。
	\begin{tableobject}{contains subobjects,
			caption = 分为子表的表格浮动体,
			label = tab:tabs,
			subobject linewidth = .45\linewidth,
		}
		\begin{subobject}{caption=Abc \& 123}{rl}
			\toprule
			abc & 123 \\
			de  & 45  \\
			f   & 6   \\
			\bottomrule
		\end{subobject}
		\begin{subobject}{caption=123 \& abc}{lr}
			\toprule
			123 & abc \\
			45  & de  \\
			6   & f   \\
			\bottomrule
		\end{subobject}
	\end{tableobject}
	
	\subsection{子图}
	可以使用\verb|figureobject|环境结合\verb|includegraphicsubobject|命令
	实现子图浮动体排版,通过\verb|<key>=<value>|设置排版参数,如图\ref{fig:figs}所示。
	\begin{figureobject}{sub,
			label   = fig:figs,
			caption = 带有子图的插图浮动体,
			graphic width = .5\linewidth,
		}
		\includegraphicsubobject[caption = 子图1, label = fig1]{example-image-a}
		\includegraphicsubobject[caption = 子图2, label = fig2]{example-image-b}
	\end{figureobject}
	
	\section{长表格}
	可以通过为\verb|easyfloats|宏包引入\verb|[longtable]|参数使用
	\verb|longtable|宏包实现长表格浮动体浮动体排版,通
	过\verb|<key>=<value>|设置排版参数,如表\ref{tab:trifun}所示。
	\begin{tableobject}{%
			caption = Trigonometric functions,
			label = tab:trifun,
			env=longtable,
			arg = {
				S[table-format=2.0, table-space-text-post=\si{\degree}] <{\si{\degree}} !\quad
				*2{S[table-format=+1.2]}
				S[table-format=+4.2]
			},
			table head = \multicolumn1{c!\quad}{$x$} & $\sin x$ & $\cos x$ & $\tan x$,
		}
		
		0  &   0.00 &  1.00 &   0.00 \\
		5  &   0.09 &  1.00 &   0.09 \\
		10  &   0.17 &  0.98 &   0.18 \\
		15  &   0.26 &  0.97 &   0.27 \\
		20  &   0.34 &  0.94 &   0.36 \\
		25  &   0.42 &  0.91 &   0.47 \\
		30  &   0.50 &  0.87 &   0.58 \\
		35  &   0.57 &  0.82 &   0.70 \\
		40  &   0.64 &  0.77 &   0.84 \\
		45  &   0.71 &  0.71 &   1.00 \\
		50  &   0.77 &  0.64 &   1.19 \\
		55  &   0.82 &  0.57 &   1.43 \\
		60  &   0.87 &  0.50 &   1.73 \\
		65  &   0.91 &  0.42 &   2.14 \\
		70  &   0.94 &  0.34 &   2.75 \\
		75  &   0.97 &  0.26 &   3.73 \\
		80  &   0.98 &  0.17 &   5.67 \\
		85  &   1.00 &  0.09 &  11.43 \\
		90  &   1.00 &  0.00 & \pminfty \\
	\end{tableobject}
	
	有关\verb|easyfloats|宏包的使用细节,请参阅其宏包使用说明书及其使用示
	例。本文档同时提供该宏包的仓库打包下载。       
	
\end{document}

%%% Local Variables:
%%% mode: latex
%%% TeX-master: t
%%% End:
