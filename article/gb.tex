%!TEX program = xelatex
\documentclass[cn,hazy,blue,14pt,screen]{elegantnote}
\makeatletter
\def\pgfutil@insertatbegincurrentpagefrombox#1{%
	\edef\pgf@temp{\the\wd\pgfutil@abb}%
	\global\setbox\pgfutil@abb\hbox{%
		\unhbox\pgfutil@abb%
		\hskip\dimexpr2in-2\hoffset-\pgf@temp\relax% changed
		#1%
		\hskip\dimexpr-2in-2\hoffset\relax% new
	}%
}
\makeatother
\title{GB18030编码研究}

\author{\href{http://www.fmddlmyy.cn/text30.html}{fmddlmyy}}
\institute{GBK、GB18030与Unicode的映射}

\version{1.0}
\date{\zhtoday}

\usepackage{array}

\begin{document}

\maketitle

\centerline{
	\includegraphics[width=0.2\textwidth]{logo.png}
}


GB18030有两个版本:GB18030-2000和GB18030-2005。在本文中,没有指明版本的GB18030是指GB18030-2005。本文讨论了以下问题:

\begin{enumerate}
\item
  GB2312有682个图形符号,都放在1区。GBK的1区有717个图形符号,5区有166个图形符号,一共有883个图形符号。GB18030的1区有728个图形符号,5区还是166个符号。那么,GBK的1区在GB2312基础上增加了哪35个符号?GB18030又增加了哪些符号?
\item
  GBK支持21003个汉字与883个图形符号,一共21886个字符。这21886个字符究竟是哪些字符?这21886个字符的编码在GB18030中有什么变化?
\item
  GB18030是怎样映射Unicode的全部0x110000个码位的?
\item
  GB18030-2000和GB18030-2005在字汇上有什么区别,在编码上有什么区别?
\item
  GB18030-2005的双字节区中有2067个码位被映射到Unicode
  BMP的PUA。这些码位有什么规律?这些码位中定义了多少字符?其实这2067个码位中只定义了24个字符。
\item
  GBK的21886个字符中有95个字符被映射到Unicode
  BMP的PUA。在GB18030中这95个字符的编码有哪些变化?哪些字符保持了原来的编码?
\item
  GBK的23940个码位中有多少码位被映射到Unicode
  BMP的PUA?在GB18030中这些码位的编码有什么变化?
\end{enumerate}

在讨论这些问题前,我们先约定一下码位空间的表示方法。


\section{码位空间}


\subsection{约定}

GBK是双字节编码,每个字符用两个字节表示。GB18030是多字节字符集,它的字符可以用一个、两个或四个字节表示。码位空间由各字节的范围确定。例如:GB18030的四字节字符码位空间是:

\begin{itemize}

\item  第一字节在0x81\textasciitilde0xFE之间
\item  第二字节在0x30\textasciitilde0x39之间
\item  第三字节在0x81\textasciitilde0xFE之间
\item  第四字节在0x30\textasciitilde0x39之间
\end{itemize}

为了表述方便,我们用0x81308130\textasciitilde0xFE39FE39表示这个码位空间。也就是说:在本文中0x81308130到0xFE39FE39所指的并\textcolor{red}{\textbf{不}}是从0x81308130到0xFE39FE39的连续2097773834(0xFE39FE39-0x81308130+1)个字节。在本文中,0x81308130\textasciitilde0xFE39FE39所指的是编码的各字节在对应范围内的码位空间,这个码位空间的码位数目是:

(0xFE-0x81+1)*(0x39-0x30+1)*(0xFE-0x81+1)*(0x39-0x30+1)=126*10*126*10=1587600

同理,0xB0A1\textasciitilde0xF7FE代表的码位空间是第一字节在0xB0\textasciitilde0xF7之间,第二字节在0xA1\textasciitilde0xFE之间的所有码位。这个码位空间的码位数目是:

(0xF7-0xB0+1)*(0xFE-0xA1+1)=72*94=6768

这个码位空间就是GBK和GB18030的2区,在这6768个码位中定义了6763个字符。

本文用\textasciitilde 表示上述码位空间,用-表示一般的范围,即:

\begin{itemize}

\item
  0xA1A1\textasciitilde0xA9FE
  表示第一字节在0xA1到0xA9之间,第二字节在0xA1\textasciitilde0xFE之间的846((0xA9-0xA1+1)*(0xFE-0xA1+1)=9*94)个码位。
\item
  0xE000-0xF8FF 表示从0xE000-0xF8FF的连续6400(0xF8FF-0xE000+1)个码位。
\end{itemize}


\subsection{习题\label{1.2}}

读者如果已经理解了上面的约定,请完成下面两个习题:

\begin{enumerate}

\item
  习题一:求码位空间0x8140\textasciitilde0xFE7E的码位数目。
\item
  习题二:求码位空间0x8180\textasciitilde0xFEFE的码位数目。
\end{enumerate}


\subsection{答案}

以下是习题\hyperlink{1.2}{1.2}的答案:

\begin{enumerate}

\item
  习题一:(0xFE-0x81+1)*(0x7E-0x40+1)=126*63=7938
\item
  习题二:(0xFE-0x81+1)*(0xFE-0x80+1)=126*127=16002
\end{enumerate}

GB18030双字节字符的码位空间就是0x8140\textasciitilde0xFE7E和0x8180\textasciitilde0xFEFE,双字节字符的码位数目是7938+16002=23940。0x8140\textasciitilde0xFE7E和0x8180\textasciitilde0xFEFE也是GBK的全部码位空间。GBK在这23940个码位中定义了21886个字符。

\hypertarget{gbkux56deux987e}{%
\section{GBK回顾}\label{gbkux56deux987e}}

\hypertarget{ux7b80ux4ecb}{%
\subsection{简介}\label{ux7b80ux4ecb}}

GBK是双字节\hyperref{1.2}{linktext}编码方案。它的码位空间就是前面所说的0x8140\textasciitilde0xFE7E和0x8180\textasciitilde0xFEFE,一共23940个码位。在这23940个码位上定义了21886个字符,包括21003个汉字和883个图形符号。\href{text24.html}{《Unicode、GB2312、GBK和GB18030中的汉字》}详细讨论了这21003个汉字。
本文的第3节会讨论GB2312、GBK和GB18030的图形符号。

GBK的码位空间可以划分为以下区域:

类别

区名

码位范围

码位数

字符数

符号区

1区

0xA1A1\textasciitilde0xA9FE

846

717

5区

0xA840\textasciitilde0xA97E和0xA880\textasciitilde0xA9A0

192

166

汉字区

2区

0xB0A1\textasciitilde0xF7FE

6768

6763

3区

0x8140\textasciitilde0xA07E和0x8180\textasciitilde0xA0FE

6080

6080

4区

0xAA40\textasciitilde0xFE7E和0xAA80\textasciitilde0xFEA0

8160

8160

用户自定义区

用户区1

0xAAA1\textasciitilde0xAFFE

564

用户区2

0xF8A1\textasciitilde0xFEFE

658

用户区3

0xA140\textasciitilde0xA77E和0xA180\textasciitilde0xA7A0

672

\hypertarget{gbkux5b57ux7b26ux4e0eunicodeux7684ux6620ux5c04}{%
\subsection{GBK字符与Unicode的映射}\label{gbkux5b57ux7b26ux4e0eunicodeux7684ux6620ux5c04}}

我制作了一个Excel文件:\href{samples/gbk.zip}{附件1}。这个文件包含3张表格:

\begin{enumerate}

\item
  按照GBK编码排序的GBK全部21886字符码表。这个表格有3列:字符、GBK编码、Unicode编码。
\item
  按照Unicode编码排序的GBK全部21886字符码表。这个表格有3列:字符、Unicode编码、GBK编码。
\item
  从按Unicode编码排序的表格中,很容易找到被映射到PUA(0xE000-0xF8FF)的字符。GBK的21886个字符中有95个字符属于PUA。第三张表格列出了这95个字符(A列)的GBK编码(B列)、Unicode编码(C列)以及这些字符在GB18030中对应的Unicode编码(D列)。\\
  其中D列可能不太容易理解,我再解释一下。GB18030是兼容GBK的,所以这些字符的GBK编码和GB18030编码是相同的。例如的GBK编码和GB18030编码都是0xA8BF。但是在GBK和GB18030中,被映射到不同的Unicode码位。在GBK中,0xA8BF被映射到Unicode的0xE7C8。在Unicode中,码位0xE7C8是一个PUA码位,保留给用户使用。在GB18030中,0xA8BF被映射到Unicode的0x01F9。在Unicode中,码位0x01F9属于``拉丁字母扩充-B''这个Block,这个码位定义的字符是``带抑音符的拉丁文小写字母N'',字形就是。
\end{enumerate}

\hypertarget{gbkux7801ux4f4dux4e0eunicodeux7684ux6620ux5c04}{%
\subsection{GBK码位与Unicode的映射}\label{gbkux7801ux4f4dux4e0eunicodeux7684ux6620ux5c04}}

GBK的23940个码位定义了21886个字符,还有23940-21886=2054个空闲码位,这2054个码位都被映射到Unicode的PUA。在设计GBK时,GBK的21886个字符中有95个在Unicode中没有对应字符,所以这95个字符也被映射到Unicode的PUA。在GBK的23940个码位中,一共有2054+95=2149个码位被映射到PUA,对应的PUA编码是0xE000-0xE864。0xE000-0xE864就是2149个码位。这2149个码位的分配有以下规律:

%\begin{longtable}[]{@{}lll@{}}
%\toprule
%\endhead
%码位所在区域 & 码位数量 & 映射到的PUA范围\tabularnewline
%用户区1:0xAAA1\textasciitilde0xAFFE & 564 &
%0xE000-0xE233\tabularnewline
%用户区2:0xF8A1\textasciitilde0xFEFE & 658 &
%0xE234-0xE4C5\tabularnewline
%用户区3:0xA140\textasciitilde0xA77E和A180-A7A0 & 672 &
%0xE4C6-0xE765\tabularnewline
%符号区(1区和5区)的170个空闲码位 & 170 & 0xE766-0xE80F\tabularnewline
%2区的5个空闲码位:0xD7FA-0xD7FE & 5 & 0xE810-0xE814\tabularnewline
%4区的80个Unicode当时没有定义的字符:FE50-FE7E和FE80-FEA0 & 80 &
%0xE815-0xE864\tabularnewline
%\bottomrule
%\end{longtable}

\href{samples/gbkcode.zip}{附件2}包含两张表格:

\begin{enumerate}

\item
  23940个GBK码位与Unicode的映射。两组数据分别按GBK和Unicode排序。
\item
  2149个映射到PUA的码位,按Unicode顺序排列。
\end{enumerate}

\hypertarget{gb18030ux7f16ux7801}{%
\section{GB18030编码}\label{gb18030ux7f16ux7801}}

\hypertarget{ux6982ux8ff0}{%
\subsection{概述}\label{ux6982ux8ff0}}

GB18030是多字节字符集,它的字符可以用一个、两个或四个字节表示。GB18030的码位定义如下:

%\begin{longtable}[]{@{}llll@{}}
%\toprule
%\endhead
%字节数 & 码位空间 & 码位数 & 字符数\tabularnewline
%单字节 & 0x00\textasciitilde0x7F & 128 & 128\tabularnewline
%双字节 & 0x8140\textasciitilde0xFE7E和0x8180\textasciitilde0xFEFE &
%23940 & 21897\tabularnewline
%四字节 & 0x81308130\textasciitilde0xFE39FE39 & 1587600 &
%54531\tabularnewline
%\bottomrule
%\end{longtable}

GB18030有128+23940+1587600=1611668个码位。Unicode的码位数目是0x110000(1114112),少于GB18030。所以,GB18030有足够的空间映射Unicode的所有码位。

GB18030的1611668个码位目前定义了128+21897+54531=76556个字符。Unicode
5.0定义了99089个字符。

\hypertarget{ux8bbeux8ba1ux601dux8def}{%
\subsection{设计思路}\label{ux8bbeux8ba1ux601dux8def}}

GB18030编码可以分为:单字节部分、双字节部分和四字节部分。单字节部分与Unicode的0x00-0x7f完全相同。双字节部分与GBK有两点差异:

\begin{enumerate}

\item
  在1区增加了11个字符。这样1区就有717+11=728个字符。增加的11个字符是:一个欧元符号(0xA2E3)和10个竖排标点符号(0xA6D9-0xA6DF、0xA6EC-0xA6ED和0xA6F3)。
\item
  原来因为Unicode没有收录而映射到PUA的字符中的部分字符被新版本的Unicode收录,所以将这些字符映射到非PUA的码位。
\end{enumerate}

Unicode的BMP一共有65536个码位。其中代理区(0xD800-0xDFFF)有2048个码位,这2048个码位是不能定义字符的。GB18030的单字节部分映射了128个码位,GB18030的双字节部分映射了23940个码位。还剩下65536-2048-128-23940=39420个码位。

GB18030将这39420个码位顺序映射到从0x81308130开始的码位空间。GB18030将Unicode的16个辅助平面(0x10000-0x10FFFF,一共1048576个码位)顺序映射到从0x90308130开始的码位空间。GB18030四字节部分中只有这两个区域定义了字符,其它空间都是保留区和自定义区。本文的第3节和第4节还会详细讨论GB18030的双字节和四字节部分。

GB18030的设计思路可以概括到以下几点:

\begin{enumerate}

\item
  单字节部分与Unicode一致。
\item
  双字节部分与GBK兼容。适当调整一些字符与Unicode的映射。这些字符原来因为Unicode没有收录而被映射到PUA,现在因为Unicode已经收录而调整到非PUA的Unicode码位。
\item
  将Unicode
  BMP部分还没有映射的39420个码位顺序映射到从0x81308130开始的四字节部分。
\item
  将Unicode
  BMP以外的16个辅助平面映射到39420个码位顺序映射到从0x90308130开始的四字节部分。
\end{enumerate}

在GB18030目前定义的76556个字符中,只有24个字符被定义到Unicode的PUA区。这24个字符包括1区的10个竖排标点符号(0xA6D9-0xA6DF、0xA6EC-0xA6ED和0xA6F3)和4区的14个汉字(0xFE51、0xFE52、0xFE53、0xFE59、0xFE61、0xFE66、0xFE67、0xFE6C、0xFE6D、0xFE76、0xFE7E、0xFE90、0xFE91、0xFEA0)。4区的14个汉字在Unicode
5.0中其实也可以找到非PUA的编码,详见\href{text24.html}{《Unicode、GB2312、GBK和GB18030中的汉字》}。但按照GB18030,它们还是应该映射到PUA码位。

\hypertarget{gb18030-2000ux548cgb18030-2005ux7684ux533aux522bux53caux4ee5ux540eux7248ux672c}{%
\subsection{GB18030-2000和GB18030-2005的区别及以后版本}\label{gb18030-2000ux548cgb18030-2005ux7684ux533aux522bux53caux4ee5ux540eux7248ux672c}}

GB18030-2005与GB18030-2000的编码体系结构是完全相同的。GB18030-2005相对于GB18030-2000主要有以下变化:

\begin{enumerate}

\item
  在四字节字符表中增加CJK统一汉字扩充B和已经在GB13000中编码的我国少数民族文字字符的字形。其实GB18030-2000已经映射了这些码位,但GB18030-2000没有给出这些字符的字形。
\item
  调整字符的编码。
\end{enumerate}

其中的编码调整比较有意思。的GB18030编码是0xA8BC,在Unicode
5.0的编码是0x1E3F。
在GB18030-2000中0xA8BC被映射到Unicode的0xE7C7,因为双字节部分没有映射0x1E3F,所以它作为BMP的未映射字符被放到四字节部分的0x8135F437。GB18030-2005将0xA8BC映射到0x1E3F,那么Unicode码位0xE7C7怎么办呢?为了最小化对原来编码的影响,设计者将Unicode码位0xE7C7映射到本来映射0x1E3F的0x8135F437。

GB18030已经映射了Unicode的所有码位,所以不管Unicode怎么变化,GB18030不过就是在现在的码位上增加一些字形而已,编码不会变化。只有现在还映射到PUA的24个字符以后可能会调整到非PUA码位。调整方法应该与的调整方法相同。

\hypertarget{gb18030ux53ccux5b57ux8282ux90e8ux5206}{%
\subsection{GB18030双字节部分}\label{gb18030ux53ccux5b57ux8282ux90e8ux5206}}

前面已经介绍过GB18030双字节部分与GBK的区别,本小节再提一些细节。前面也说过,GB18030映射了Unicode除代理区外的所有码位。所以,Unicode BMP的6400个PUA码位在GB18030中都有对应的码位。GB18030双字节部分映射了2067个PUA码位。

前面说过,GBK映射了2149个PUA码位。现在GB18030双字节部分映射了2067个PUA码位。所以有2149-2067=82个字符的映射发生了变化。GBK原来有95个字符映射到PUA,其中81个字符在GB18030中被映射到非PUA码位。余下的14个汉字就是\href{text24.html}{《Unicode、GB2312、GBK和GB18030中的汉字》}提到的那14个汉字(0xFE51、0xFE52、0xFE53、0xFE59、0xFE61、0xFE66、0xFE67、0xFE6C、0xFE6D、0xFE76、0xFE7E、0xFE90、0xFE91、0xFEA0)。\href{samples/gbk.zip}{附件1}列出了这些字符的编码变化。82个映射变化的码位,除了这81个外,还有一个就是欧元符号:GB18030编码是0xA2E3,Unicode编码是0x20AC。码位0xA2E3在GBK中被映射到0xE76C,GBK的码位0xA2E3没有定义字符。

GB18030双字节部分与Unicode的映射没有规律,只能通过查表方法映射。

\hypertarget{gb18030ux56dbux5b57ux8282ux90e8ux5206}{%
\subsection{GB18030四字节部分}\label{gb18030ux56dbux5b57ux8282ux90e8ux5206}}

GB18030四字节部分的字符可以见GB18030-2005的``表3
四字节部分的码位安排'',一共54531个字符。GB18030四字节部分的码位可以见GB18030-2005的``7.3
四字节部分字符的排列顺序''。其中定义字符的只有两个区域:

\begin{itemize}

\item
  GB18030用码位0x81308130\textasciitilde0x8439FE39共50400个码位映射该标准单字节和双字节部分没有映射过的39420个Unicode
  BMP码位。
\item
  GB18030用码位0x90308130\textasciitilde0xE339FE39共1058400个码位映射Unicode
  16个辅助平面(平面1到平面16)的65536*16=1048576个码位。
\end{itemize}

为了叙述方便,本文将0x81308130\textasciitilde0x8439FE39称作``BMP扩展部分'',将0x90308130\textasciitilde0xE339FE39称作``辅助平面部分''。GB18030四字节部分的码位空间是0x81308130\textasciitilde0xFE39FE39。第二字节有(0x39-0x30+1)=10个可能值。第三字节有(0xFE-0x81+1)=126个可能值。第四字节也是(0x39-0x30+1)=10个可能值。为了方便下面的演算,本文为这个码位空间定义几个名词:

\begin{itemize}

\item
  我们将四字节码位空间中第一字节相同的区域称作\textbf{一级区}。每个一级区有12600个码位,即:10*126*10。
\item
  我们将四字节码位空间中第一字节和第二字节相同的区域称作\textbf{二级区}。每个二级区有1260个码位,即:126*10。
\item
  我们将四字节码位空间中前三个字节相同的区域称作\textbf{三级区},每个三级区有10个码位。
\end{itemize}

四字节部分一共有(0xFE-0x81+1)=126个一级区。BMP扩展部分有4个一级区。辅助平面部分有84个一级区。还有38个一级区是保留区或自定义区。

\hypertarget{bmpux6269ux5c55ux90e8ux5206}{%
\paragraph{BMP扩展部分}\label{bmpux6269ux5c55ux90e8ux5206}}

BMP扩展部分占据四字节部分开头的4个一级区,一共有4*12600=50400个码位。这段空间的Unicode映射说起来还是很简单的,就是顺序映射单字节、双字节没有映射过的BMP码位。这些映射关系在GB18030-2000中确定下来。以后的调整(例如)只是个别字符,不会影响其它字符的位置。但是因为双字节字符已经映射过的BMP码位没有什么规律,所以造成BMP扩展部分的Unicode映射也不能用公式换算,还是要查表解决。

显然这50400个码位中只用到了39420个码位,其余码位都是保留的。出于好玩,我们来计算一下最后一个非保留码位(0xFFFF)的位置,计算过程如下:

\begin{itemize}

\item
  m1=(39420-1)/12600=3
\item
  n1=(39420-1)\%12600=1619
\item
  m2=n1/1260=1619/1260=1
\item
  n2=n1\%1260=1619\%1260=359
\item
  m3=n2/10=359/10=35
\item
  n3=n2\%10=359\%10=9
\item
  第一字节的位置是:0x81+m1=0x81+3=0x84
\item
  第二字节的位置是:0x30+m2=0x30+1=0x31
\item
  第三字节的位置是:0x81+m3=0x81+35=0xA4
\item
  第四字节的位置是:0x30+n3=0x30+9=0x39
\end{itemize}

所以Unicode编码0xFFFF映射的GB18030码位是0x8431A439。在BMP扩展部分中,0x8431A439以后的码位都是保留码位。上述计算中,/表示整除(例如5/3=1),\%表示取余(例如5\%3=2)。

\hypertarget{ux8f85ux52a9ux5e73ux9762ux90e8ux5206}{%
\paragraph{辅助平面部分}\label{ux8f85ux52a9ux5e73ux9762ux90e8ux5206}}

辅助平面部分用84个一级区(0x90308130\textasciitilde0xE339FE39)直接映射Unicode的16个辅助平面。这部分映射是可以直接用公式计算的。让我们看看怎么计算。

\begin{itemize}

\item
  从Unicode编码到GB18030编码的映射方法如下:\\

  \begin{itemize}
  
  \item
    U=Unicode编码-0x10000
  \item
    m1=U/12600
  \item
    n1=U\%12600
  \item
    m2=n1/1260
  \item
    n2=n1\%1260
  \item
    m3=n2/10
  \item
    n3=n2\%10
  \item
    第一字节b1=m1+0x90
  \item
    第二字节b2=m2+0x30
  \item
    第三字节b3=m3+0x81
  \item
    第四字节b4=n3+0x30
  \end{itemize}

  按照上述方法可以计算出0x10FFFF被映射到0xE3329A35。在辅助平面部分,0xE3329A35以后的码位都是保留码位。以上所写的算法可以很容易写成C/C++代码。对于不会编程的读者,也可以用Excel公式计算。假设Unicode编码放在单元格A12,计算方法如下:\\

  \begin{itemize}
  
  \item
    将m1放在B12,B12=INT((HEX2DEC(A12)-65536)/12600)
  \item
    将n1放在C12,C12=MOD((HEX2DEC(A12)-65536),12600)
  \item
    将m2放在D12,D12=INT(C12/1260)
  \item
    将n2放在E12,E12=MOD(C12,1260)
  \item
    将m3放在F12,F12=INT(E12/10)
  \item
    将n3放在G12,G12=MOD(E12,10)
  \item
    将第一字节放在H12,H12=DEC2HEX(B12+144)
  \item
    将第二字节放在I12,I12=DEC2HEX(D12+48)
  \item
    将第三字节放在J12,J12=DEC2HEX(F12+129)
  \item
    将第四字节放在K12,K12=DEC2HEX(G12+48)
  \end{itemize}

  \href{samples/gb18030.zip}{附件3}中有写好上述公式的Excel表格。使用函数HEX2DEC/DEC2HEX需要通过``工具-\textgreater 加载宏''钩上``分析工具库''。
\item
  从GB18030编码到Unicode编码的映射方法如下:

  \begin{itemize}
  
  \item
    设GB18030编码的四个字节依次为:b1、b2、b3、b4,则\\
    Unicode编码=0x10000+(b1-0x90)*12600+(b2-0x30)*1260+(b3-0x81)*10+b4-0x30
  \end{itemize}

  假设b1、b2、b3、b4分别放在A4、B4、C4、D4,Unicode编码放在E4,则Excel计算公式为:

  \begin{itemize}
  
  \item
    E4 =
    =DEC2HEX((HEX2DEC(A4)-144)*12600+(HEX2DEC(B4)-48)*1260+(HEX2DEC(C4)-129)*10+(HEX2DEC(D4)-48)+65536)
  \end{itemize}
\end{itemize}

\hypertarget{gb18030ux548cunicodeux7684ux6620ux5c04ux8868}{%
\subsection{GB18030和Unicode的映射表}\label{gb18030ux548cunicodeux7684ux6620ux5c04ux8868}}

\href{samples/gb18030.zip}{附件3}给出了GB18030和Unicode的映射表。这个Excel文件是在网友谢振斌先生的\href{http://www.pkucn.com/viewthread.php?tid=188399\&extra=page\%3D1}{映射表}基础上制作的,包含3张表格:

\begin{enumerate}

\item
  双字节部分23940个码位与Unicode的映射。两组数据分别按GB18030和Unicode排序。
\item
  BMP扩展部分39420个码位与Unicode的映射。两组数据分别按GB18030和Unicode排序。
\item
  辅助平面部分,GB18030编码和Unicode编码的映射公式。
\end{enumerate}

\hypertarget{gb2312gbkux548cgb18030ux4e2dux7684ux56feux5f62ux7b26ux53f7}{%
\section{GB2312、GBK和GB18030中的图形符号}\label{gb2312gbkux548cgb18030ux4e2dux7684ux56feux5f62ux7b26ux53f7}}

在研究GB18030编码的过程中,我整理了GB2312、GBK和GB18030在1区和5区的图形符号,制作了\href{samples/sym.zip}{附件4}。这个Excel文件包含3张表格:

\begin{enumerate}

\item
  GB2312的1区字符表。GBK和GB18030的1区、5区字符表。用不同颜色标注了GBK增加的35个字符和GB18030增加的11个字符。
\item
  GB2312 1区682个符号的编码。
\item
  GBK 1区717个符号的编码。
\end{enumerate}

\hypertarget{ux7ed3ux675fux8bed}{%
\section{结束语}\label{ux7ed3ux675fux8bed}}

通过本文的介绍,读者可以回答开头的问题了吗?

无论是Windows
XP还是Vista,中文(中国)区域对应的默认代码页还是GBK。我们只能设置区域,并不能设置区域对应的默认代码页。
所以在Windows世界,只要微软不愿意,GB18030就只是一张普通的代码页。目前的简体中文文档使用的编码主要是Unicode和GBK,应该没有什么文档会用GB18030保存。本文只是出于程序员的好奇而对GB18030编码所作的一些研究,希望能对同样好奇的读者有所助益。

~

%\begin{longtable}[]{@{}ll@{}}
%\toprule
%\endhead
%\begin{minipage}[t]{0.47\columnwidth}\raggedright
%\href{http://www.google.com/}{\includegraphics{http://www.google.com/logos/Logo_25wht.gif}}\strut
%\end{minipage} & \begin{minipage}[t]{0.47\columnwidth}\raggedright
%输入您的搜索字词 提交搜索表单\strut
%\end{minipage}\tabularnewline
%\begin{minipage}[t]{0.47\columnwidth}\raggedright
%~\strut
%\end{minipage} & \begin{minipage}[t]{0.47\columnwidth}\raggedright
%\begin{longtable}[]{@{}ll@{}}
%\toprule
%\endhead
%Web & www.fmddlmyy.cn\tabularnewline
%\bottomrule
%\end{longtable}\strut
%\end{minipage}\tabularnewline
%\bottomrule
%\end{longtable}

\begin{center}\rule{0.5\linewidth}{0.5pt}\end{center}

\href{index.html}{个人主页} \textbar~
\href{http://post.baidu.com/f?kw=fmddlmyy}{留言本} \textbar~
\href{http://hi.baidu.com/fmddlmyy}{我的空间} \textbar~
\href{myprog.html}{我的程序} \textbar~
\includegraphics[width=0.29167in,height=0.21875in]{logo.png}
\href{mailto:fmdd@263.net}{\nolinkurl{fmdd@263.net}}

\end{document}
