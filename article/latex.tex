% !Mode:: "TeX:UTF-8"
\documentclass[fontset=none]{ctexart}
\usepackage{geometry}
\geometry{margin=2cm}
\usepackage[all]{tcolorbox} 
\usepackage{url}
\usepackage{fontawesome}
\usepackage{zhlipsum}

%===设置本站支持的中文字体==中文字体下载:https://www.latexstudio.net/fonts/font.zip 方正字体勿商用=====
\setCJKmainfont{Source Han Serif CN}
[
UprightFont = *-Regular,
BoldFont = *-Bold,
ItalicFont = *-Regular,
BoldItalicFont = *-Bold
]
\setCJKsansfont{Noto Sans CJK SC}
[
UprightFont = *-Regular,
BoldFont = *-Bold,
ItalicFont = *-Regular,
BoldItalicFont = *-Bold
]
\setCJKmonofont{Noto Sans Mono CJK SC}
[
UprightFont = *-Regular,
BoldFont = *-Bold,
AutoFakeSlant = 0.2
]

\setmainfont{TeX Gyre Pagella}

\newcommand*{\songti}{\CJKfamily{zhsong}} % 宋体
\newcommand*{\heiti}{\CJKfamily{zhhei}} % 黑体
\newcommand*{\kaiti}{\CJKfamily{zhkai}} % 楷体
%=========================================================



%%%%%%%%%%%%%颜色%%%%%%%%%%%%%%%
\definecolor{doc}{RGB}{33,133,197}
\definecolor{ExampleFrame}{RGB}{48,109,224}

\usepackage{ascolorbox}



\definecolor{exampleborder}{HTML}{FE642E}
\definecolor{examplebg}{HTML}{CEF6EC}
\definecolor{statementborder}{rgb}{.9,0,0}
\definecolor{statementbg}{rgb}{1,1,1}
\definecolor{backcolor}{RGB}{48,108,223}
\definecolor{listcolor}{RGB}{255,255,229}
\definecolor{listred}{RGB}{199,15,44}
\ctexset{
	section = {
		format=\raggedright\bfseries\sffamily,numberformat =\rmfamily,
		beforeskip={2ex},
		afterskip={2ex plus 3pt minus 3pt}
}}

\tcbset{listing engine=minted}
\tcbset{
	docexample/.add style={bicolor,colbacklower=white}{
		fontlower=\normalsize,
		documentation minted language=latex,
		documentation minted options={
			autogobble=true,
			fontsize=\footnotesize,
		}
	}
}


\usepackage{dtk-logos}
\AtBeginDocument{\hypersetup{hidelinks}}
%%%%%%%%%%%%填写相关信息%%%%%%%%%%%%%%%%%%%%%%%%%%
\usepackage{cleveref}
\usepackage{menukeys}
\usepackage[bottom]{footmisc}
\begin{document}
	
	
	\begin{titlepage}
		\thispagestyle{empty}
		
		\vspace*{3cm}
		\begin{center}
			\huge
			\ascboxA{\textbf{ascolorbox 宏包使用手册 1.0.3}}
			
		\end{center}
		
		\bigskip\bigskip
		
		\begin{center}
			\includegraphics[height=1cm]{latexstudio} 
			
			倾情制作 
			
%			\includegraphics[width=6cm]{gongzhonghao}
			
			扫码关注我们公众号!
			
			
			\begin{minipage}{9.5cm}
				\begin{ascolorbox17}{相关说明}
					\color{listred}
					宏包(日文)地址:
					
					\url{https://github.com/Yasunari/ascolorbox}
					
					本文对宏包进行了中文化的修订,大家可下载试用。
				\end{ascolorbox17}
			\end{minipage}
		\end{center}
		
		
		
		
		
	\end{titlepage}
	\newpage 
	%\maketitle
	
	ascolorbox 参考tcolorbox宏包手册,充分利用该宏包的 \verb|\DeclareTColorBox|机制,提供了方便
	设计、创建各种框的参数。 ascolorbox里的框大多都定义成环境,语法方面遵循普通tcolorbox的用法。
	使用的时候只需 \cs{usepackage\{ascolorbox\}}即可。
	
	
	
	
	
	%\newpage
	%目录
	%\tableofcontents
	
	\section{使用说明}
	这是一个简单的盒子,四周只有一个单线框。根据 \verb|itembbox.sty| 提供的 itemsquarebox 样式创建的。
	\oarg{subtitle} 是可选参数,用于指定副标题,并且可以用好\oarg{thickness}来指定边框的粗细,边框的默认粗细为 0.5pt。
	老版本是不支持换页的,新版本已经支持分页了。由于使用了  \docColor{empty skin} 参数,因此 \docColor{colback = gray} 要多加注意。
	
	\begin{dispExample}
		\begin{simplesquarebox}[子标题]{标题}
			\zhlipsum[1]
		\end{simplesquarebox}
	\end{dispExample}
	
	\begin{dispExample}
		\begin{simplesquarebox}[子标题]{标题}[1.5][colback=gray]
			\zhlipsum[1]
		\end{simplesquarebox}
	\end{dispExample}
	
	\begin{dispListing*}{colback=listcolor,colframe=listred}
		\begin{practicebox}{<title>}[<options>]
			environment content
		\end{practicebox}
	\end{dispListing*}
	
	练习/问题框。可以在 \docColor{options} 中进行选项设置,该框支持分页。
	\begin{dispExample}
		\begin{practicebox}{欧拉的遗产}
			\zhlipsum[1]
		\end{practicebox}
	\end{dispExample}
	
	
	\begin{dispListing*}{colback=listcolor,colframe=listred}
		\begin{ascolorbox1}[<subtitle>]{<title>}[<options>]
			environment content
		\end{ascolorbox1}
	\end{dispListing*}
	
	这是 tcolorbox 手册自带框的黑白版本,支持分页。 \oarg{subtitle} 是选择项, \oarg{option} 可以自动加定义。
	
	\begin{dispExample}
		\begin{ascolorbox1}[子标题]{标题}
			\zhlipsum[1]
		\end{ascolorbox1}
	\end{dispExample}
	
	
	
	\begin{dispListing*}{colback=listcolor,colframe=listred}
		\begin{ascolorbox2}{<title>}[<options>]
			environment content
		\end{ascolorbox2}
	\end{dispListing*}
	
	这是 tcolorbox 自带手册中的框的黑白版本,支持分页。 \docColor{option} 可以自动加定义。
	
	\begin{dispExample}
		\begin{ascolorbox2}{标题}
			\zhlipsum[1]
		\end{ascolorbox2}
	\end{dispExample}
	
	\begin{dispExample}
		\begin{ascolorbox2}{标题}[colframe=red!50!white, coltext=red!50!black]
			\zhlipsum[1]
		\end{ascolorbox2}
	\end{dispExample}
	
	\begin{dispListing*}{colback=listcolor,colframe=listred}
		\begin{ascolorbox3}{<title>}[<color>][<option>]
			environment content
		\end{ascolorbox3}
	\end{dispListing*}
	
	这是 tcolorbox 手册自带框的黑白版本。\docColor{color} 可以用来修改框颜色,\docColor{option} 可以自动加定义。
	
	\begin{dispExample}
		\begin{ascolorbox3}{标题}
			\zhlipsum[1]
		\end{ascolorbox3}
	\end{dispExample}
	
	
	\begin{dispListing*}{colback=listcolor,colframe=listred}
		\begin{ascolorbox3}{标题}[orange][coltitle=orange!50!black]
			\zhlipsum[1]
		\end{ascolorbox3}
	\end{dispListing*}
	
	
	\begin{ascolorbox3}{标题}[orange][coltitle=orange!50!black]
		\zhlipsum[1]
	\end{ascolorbox3}
	
	
	
	
	
	
	\begin{dispListing*}{colback=listcolor,colframe=listred}
		\begin{ascolorbox4}[<subtitle>]{<title>}[<length>][<option>]
			environment content
		\end{ascolorbox4}
	\end{dispListing*}
	
	这是 ascolorbox 原创的样式,  可以通过修改\oarg{length}来修改实线与虚线的宽度。四个角的正方形和圆弧也会跟着这一个值进行变化,其默认宽度为 3,如果在双栏环境里,推荐改为 2 版面看起来更加整洁。
	
	\begin{dispExample}
		\begin{ascolorbox4}[子标题]{标题}
			\zhlipsum[1]
		\end{ascolorbox4}
	\end{dispExample}
	
	\begin{dispExample}
		\begin{ascolorbox4}[子标题]{标题}[2]
			\zhlipsum[1]
		\end{ascolorbox4}
	\end{dispExample}
	
	
	
	\begin{dispListing*}{colback=listcolor,colframe=listred}
		\begin{ascolorbox5}[<subtitle>]{<title>}[<color>][<option>]
			environment content
		\end{ascolorbox5}
	\end{dispListing*}
	
	这是 ascolorbox 原创的样式,  可以通过修改\oarg{color}来修改颜色。
	
	\begin{dispExample}
		\begin{ascolorbox5}[子标题]{标题}
			\zhlipsum[1]
		\end{ascolorbox5}
	\end{dispExample}
	
	可在 \oarg{option} 中指定框架颜色,标题颜色和字符颜色。
	
	\begin{dispExample}
		\begin{ascolorbox5}[子标题]{标题}[black!50!white][coltext=black!10!white]
			\zhlipsum[1]
		\end{ascolorbox5}
	\end{dispExample}
	
	
	\begin{dispListing*}{colback=listcolor,colframe=listred}
		\begin{ascolorbox8}{<title>}[<option>]
			environment content
		\end{ascolorbox8}
	\end{dispListing*}
	
	这是 tcolorbox 的黑白版本,  可以通过修改\oarg{option}来修改设置。
	
	\begin{dispExample}
		\begin{ascolorbox8}{标题}
			\zhlipsum[1]
		\end{ascolorbox8}
	\end{dispExample}
	
	
	\begin{dispListing*}{colback=listcolor,colframe=listred}
		\begin{ascolorbox9}{<title>}[<number>][<option>]
			environment content
		\end{ascolorbox5}
	\end{dispListing*}
	
	这是 ascolorbox 原创的样式,  可以通过\oarg{number}来指定重复小球的个数,默认数值为3。若是在双栏排版的时候推荐修改数值为2。
	
	\begin{dispExample}
		\begin{ascolorbox9}{标题}
			\zhlipsum[1]
		\end{ascolorbox9}
	\end{dispExample}
	
	
	
	
	\begin{dispListing*}{colback=listcolor,colframe=listred}
		\begin{ascolorbox10}[<subtitle>]{<title>}[<thickness>][<option>]
			environment content
		\end{ascolorbox10}
	\end{dispListing*}
	
	只有底部和顶部的框线,可以通过 \oarg{thickness} 来修改线的粗细,默认值是 0.8。
	
	\begin{dispExample}
		\begin{ascolorbox10}[子标题]{标题}
			\zhlipsum[1]
		\end{ascolorbox10}
	\end{dispExample}
	
	
	框的左右缩进宽度为2mm,如果想增加缩进宽度,可以通过修改 \docColor{/tcb/
		enlarge left by} 和 \docColor{/tcb/enlarge right by} 来调整缩进的宽度。选项可以在保证框的宽度的同时改变左右边距,因此用户可以根据需求调整。
	
	\begin{dispExample}
		\begin{ascolorbox10}[子标题]{标题}[1][width=\linewidth-4cm,enlarge right by=2cm, enlarge left by=2cm]
			\zhlipsum[1]
		\end{ascolorbox10}
	\end{dispExample}
	
	
	\begin{dispListing*}{colback=listcolor,colframe=listred}
		\begin{ascolorbox11}[<subtitle>]{<title>}[<length>][<option>]
			environment content
		\end{ascolorbox11}
	\end{dispListing*}
	
	只有底部和顶部的框线,可以通过 \oarg{length} 来调整四角的正方形的尺寸,默认值是4pt,双栏排版时,推荐设置小一些。
	
	\begin{dispExample}
		\begin{ascolorbox11}[子标题]{标题}
			\zhlipsum[1]
		\end{ascolorbox11}
	\end{dispExample}
	
	
	
	\begin{dispListing*}{colback=listcolor,colframe=listred}
		\begin{ascolorbox12}[<subtitle>]{<title>}[<option>]
			environment content
		\end{ascolorbox12}
	\end{dispListing*}
	
	这是 ascolorbox 的原创样式。
	
	\begin{dispExample}
		\begin{ascolorbox12}[子标题]{标题}
			\zhlipsum[1]
		\end{ascolorbox12}
	\end{dispExample}
	
	
	\begin{dispListing*}{colback=listcolor,colframe=listred}
		\begin{ascolorbox13}[<subtitle>]{<title>}[<option>]
			environment content
		\end{ascolorbox13}
	\end{dispListing*}
	
	这是 tcolorbox自带的样式,这里是其黑白版本。
	
	\begin{dispExample}
		\begin{ascolorbox13}[子标题]{标题}
			\zhlipsum[1]
		\end{ascolorbox13}
	\end{dispExample}
	
	
	
	\begin{dispListing*}{colback=listcolor,colframe=listred}
		\begin{ascolorbox14}{<title>}[<option>]
			environment content
		\end{ascolorbox14}
	\end{dispListing*}
	
	用 overlay 制作的笔记本的风格的框,环的位置为其高度的四分点上,当然文本内容不能太少,这样环会溢出来版式,不好看了。
	
	\begin{dispExample}
		\begin{ascolorbox14}{标题}
			\zhlipsum[1-2]
		\end{ascolorbox14}
	\end{dispExample}
	
	
	
	
	
	
	
	\begin{dispListing*}{colback=listcolor,colframe=listred}
		\begin{ascolorbox15}{<title>}[<option>]
			environment content
		\end{ascolorbox15}
	\end{dispListing*}
	
	这是两页笔记本的风格框,环位置为其高度的四分点上,当然文本内容不能太少,这样环会溢出来版式,不好看。
	
	\begin{dispExample}
		\begin{ascolorbox15}{标题}
			\zhlipsum[1]
		\end{ascolorbox15}
	\end{dispExample}
	
	
	
	
	
	
	
	\begin{dispListing*}{colback=listcolor,colframe=listred}
		\begin{ascolorbox16}{<title>}[<option>]
			environment content
		\end{ascolorbox16}
	\end{dispListing*}
	
	这是底部和顶部有框线的,这个样式与 ascolorbox10 类似的样式。
	
	\begin{dispExample}
		\begin{ascolorbox16}{标题}
			\zhlipsum[1]
		\end{ascolorbox16}
	\end{dispExample}
	
	
	
	
	\begin{dispListing*}{colback=listcolor,colframe=listred}
		\begin{ascolorbox17}[<subtitle>]{<title>}[<color>][<option>]
			environment content
		\end{ascolorbox17}
	\end{dispListing*}
	
	这是一个括号样式的框,可以通过 \oarg{color} 来调整线条的颜色,如果颜色设置为白色,则只包括括号部分。 
	
	\begin{dispExample}
		\begin{ascolorbox17}[子标题]{标题}
			\zhlipsum[1]
		\end{ascolorbox17}
	\end{dispExample}
	
	
	\begin{dispExample}
		\begin{ascolorbox17}[子标题]{标题}[white]
			\zhlipsum[1]
		\end{ascolorbox17}
	\end{dispExample}
	
	
	
	
	
	
	\begin{dispListing*}{colback=listcolor,colframe=listred}
		\begin{ascolorbox18}{<title>}[<option>]
			environment content
		\end{ascolorbox18}
	\end{dispListing*}
	
	这是tcolorbox 手册自带的样式,参考其修改的黑白版本。
	
	\begin{dispExample}
		\begin{ascolorbox18}{标题}
			\zhlipsum[1]
		\end{ascolorbox18}
	\end{dispExample}
	
	
	稍作调整,即可调整其标题的位置。
	
	
	\begin{dispExample}
		\begin{ascolorbox18}{标题}[attach boxed title to bottom left={xshift=1.2cm}]
			\zhlipsum[1]
		\end{ascolorbox18}
	\end{dispExample}
	
	
	
	
	
	
	
	
	
	
	
	
	\begin{dispListing*}{colback=listcolor,colframe=listred}
		\begin{ascolorbox19}[<subtitle>]{<title>}[<length>][<option>]
			environment content
		\end{ascolorbox19}
	\end{dispListing*}
	
	这是 ascolorbox 原创的样式,可以通过 \oarg{length} 来调整两根线的距离,默认值是2pt。左上部分为 \verb|black!40!white|,正方形部分为 \verb|black!70!white|,右下部分为黑色。
	
	\begin{dispExample}
		\begin{ascolorbox19}[子标题]{标题}
			\zhlipsum[1]
		\end{ascolorbox19}
	\end{dispExample}
	
	\begin{dispExample}
		\begin{ascolorbox19}[子标题]{标题}[3]
			\zhlipsum[1]
		\end{ascolorbox19}
	\end{dispExample}
	
	\section{ascbox 使用方法}
	
	\cs{ascbox} 是根据 \cs{tcbox}定义的框,用来设置小标题是非常方便的。 由于 \cs{DeclareTCBox} 主要用于 \cs{ascbox},因此可以设置很多选项。 大多 \cs{ascbox}可以通过在参数部分添加\verb|*|来反转颜色。
	\begin{dispExample*}{sidebyside}
		\ascboxA{This is my title} 
		
		\ascboxB{This is my title}
	\end{dispExample*}
	
	
	通过在 \verb|{type}| 中指定 A 到 E 之一可以更改输出格式。 默认输出格式为 A。 您可以通过在 \verb|{type}| 之前添加 \verb|*|来反转色调。
	\begin{dispExample*}{sidebyside}
		\ascboxB[B]{This is my title}
		
		\ascboxB*[B]{This is my title} 
	\end{dispExample*}
	
	
	
	\begin{dispExample*}{sidebyside}
		\ascboxB[C]{This is my title}
		
		\ascboxB*[C]{This is my title}
	\end{dispExample*}
	\begin{dispExample*}{sidebyside}
		\ascboxB[D]{This is my title}
		
		\ascboxB*[D]{This is my title}
	\end{dispExample*}
	\begin{dispExample*}{sidebyside}
		\ascboxB[E]{This is my title}
		
		\ascboxB*[E]{This is my title}
	\end{dispExample*}
	
	
	在 \verb|{option}| 之后,可以用\verb|*|删除标题的下划线。 因此,如果添加两个\verb|*|,则输出将为“颜色反转+没有下划线”。
	\begin{dispExample*}{sidebyside}
		\ascboxB**{This is my title} 
	\end{dispExample*}
	
	如果要取消下划线、却又不要反转色调,则必须大括号指定省略\oarg{options}。
	\begin{dispExample*}{sidebyside}
		\ascboxB[A][]*{This is my title}
		
		\ascboxB*[C][]*{This is my title}
	\end{dispExample*}
	
	
	
	\begin{dispListing*}{colback=listcolor,colframe=listred}
		\ascboxC<star>[<type>][<options>]<star>{<title>}
	\end{dispListing*}
	
	
	\begin{dispExample*}{sidebyside}
		\ascboxC{This is my title}
		
		\ascboxC*{This is my title}
	\end{dispExample*}
	
	
	
	\begin{dispExample*}{sidebyside}
		\ascboxC[B]{This is my title}
		
		\ascboxC*[B]{This is my title}
	\end{dispExample*}
	
	
	
	\begin{dispExample*}{sidebyside}
		\ascboxC[C]{This is my title}
		
		\ascboxC*[C]{This is my title} 
	\end{dispExample*}
	
	
	
	\begin{dispExample*}{sidebyside}
		\ascboxC[D]{This is my title}
		
		\ascboxC*[D]{This is my title}
	\end{dispExample*}
	
	
	\begin{dispExample*}{sidebyside}
		\ascboxC[E]{This is my title}
		
		\ascboxC*[E]{This is my title}
	\end{dispExample*}
	
	
	
	
	\begin{dispListing*}{colback=listcolor,colframe=listred}
		\ascboxD<star>[<type>][<options>]<star>{<title>}
	\end{dispListing*}
	
	\begin{dispExample*}{sidebyside}
		\ascboxD{This is my title}
		
		\ascboxD*{This is my title}
	\end{dispExample*}
	
	
	\begin{dispExample*}{sidebyside}
		\ascboxD[B]{This is my title}
		
		\ascboxD*[B]{This is my title}
	\end{dispExample*}
	
	
	
	\begin{dispExample*}{sidebyside}
		\ascboxD[C]{This is my title}
		
		\ascboxD*[C]{This is my title}
	\end{dispExample*}
	
	
	
	\begin{dispExample*}{sidebyside}
		\ascboxD[D]{This is my title}
		
		\ascboxD*[D]{This is my title}
	\end{dispExample*}
	
	
	\begin{dispExample*}{sidebyside}
		\ascboxD[E]{This is my title}
		
		\ascboxD*[E]{This is my title}
	\end{dispExample*}
	
	\begin{dispListing*}{colback=listcolor,colframe=listred}
		\ascboxE<star>[<type>][<options>]<star>{<title>}
	\end{dispListing*}
	
	\begin{dispExample*}{sidebyside}
		\ascboxE{This is my title}
		
		\ascboxE*{This is my title}
	\end{dispExample*}
	
	
	\begin{dispExample*}{sidebyside}
		\ascboxE[B]{This is my title}
		
		\ascboxE*[B]{This is my title}
	\end{dispExample*}
	
	
	
	\begin{dispExample*}{sidebyside}
		\ascboxE[C]{This is my title}
		
		\ascboxE*[C]{This is my title}
	\end{dispExample*}
	
	
	
	\begin{dispExample*}{sidebyside}
		\ascboxE[D]{This is my title}
		
		\ascboxE*[D]{This is my title}
	\end{dispExample*}
	
	
	\begin{dispExample*}{sidebyside}
		\ascboxE[E]{This is my title}
		
		\ascboxE*[E]{This is my title}
	\end{dispExample*}
	
	
	\begin{dispListing*}{colback=listcolor,colframe=listred}
		\ascboxF<star>[<type>][<options>]<star>{<title>}
	\end{dispListing*}
	
	\begin{dispExample*}{sidebyside}
		\ascboxF{This is my title}
		
		\ascboxF*{This is my title}
	\end{dispExample*}
	
	
	\begin{dispExample*}{sidebyside}
		\ascboxF[B]{This is my title}
		
		\ascboxF*[B]{This is my title}
	\end{dispExample*}
	
	
	\begin{dispListing*}{colback=listcolor,colframe=listred}
		\ascboxG<star>[<type>][<options>]<star>{<title>}
	\end{dispListing*}
	
	\begin{dispExample*}{sidebyside}
		\ascboxG{This is my title}
		
		\ascboxG*{This is my title}
	\end{dispExample*}
	
	
	\begin{dispListing*}{colback=listcolor,colframe=listred}
		\ascboxH<star>[<type>][<options>]<star>{<title>}
	\end{dispListing*}
	
	\begin{dispExample*}{sidebyside}
		\ascboxH{This is my title}
		
		\ascboxH*{This is my title}
	\end{dispExample*}
	
	
	\begin{dispExample*}{sidebyside}
		\ascboxH[B]{This is my title}
		
		\ascboxH*[B]{This is my title}
	\end{dispExample*}
	
	
	\begin{dispListing*}{colback=listcolor,colframe=listred}
		\ascboxI<star>[<type>][<options>]<star>{<title>}
	\end{dispListing*}
	
	\begin{dispExample*}{sidebyside}
		\ascboxI{This is my title}
		
		\ascboxI*{This is my title}
	\end{dispExample*}
	
	
	\begin{dispExample*}{sidebyside}
		\ascboxI[B]{This is my title}
		
		\ascboxI*[B]{This is my title}
	\end{dispExample*}
	
	
	
	\begin{dispExample*}{sidebyside}
		\ascboxI[C]{This is my title}
		
		\ascboxI*[C]{This is my title}
	\end{dispExample*}
	
	
	
	\begin{dispExample*}{sidebyside}
		\ascboxI[D]{This is my title}
		
		\ascboxI*[D]{This is my title}
	\end{dispExample*}
	
	
	\begin{dispExample*}{sidebyside}
		\ascboxI[E]{This is my title}
		
		\ascboxI*[E]{This is my title}
	\end{dispExample*}
	
	\begin{dispExample*}{sidebyside}
		\ascboxI[F]{This is my title}
		
		\ascboxI*[F]{This is my title}
	\end{dispExample*}
	
	\begin{dispExample*}{sidebyside}
		\ascboxI[G]{This is my title}
		
		\ascboxI*[G]{This is my title}
	\end{dispExample*}
	
	\begin{dispExample*}{sidebyside}
		\ascboxI[H]{This is my title}
		
		\ascboxI*[H]{This is my title}
	\end{dispExample*}
	
	\begin{dispListing*}{colback=listcolor,colframe=listred}
		\ascboxJ<star>[<type>][<options>]<star>{<title>}
	\end{dispListing*}
	
	\begin{dispExample*}{sidebyside}
		\ascboxJ{This is my title}
		
		\ascboxJ*{This is my title}
	\end{dispExample*}
	
	\begin{dispExample*}{sidebyside}
		\ascboxJ[B]{This is my title}
		
		\ascboxJ*[B]{This is my title}
	\end{dispExample*}
	
	
	
	\begin{dispExample*}{sidebyside}
		\ascboxJ[C]{This is my title}
		
		\ascboxJ*[C]{This is my title}
	\end{dispExample*}
	
	
	
	\begin{dispExample*}{sidebyside}
		\ascboxJ[D]{This is my title}
		
		\ascboxJ*[D]{This is my title}
	\end{dispExample*}
	
	
	\begin{dispExample*}{sidebyside}
		\ascboxJ[E]{This is my title}
		
		\ascboxJ*[E]{This is my title}
	\end{dispExample*}
	
	\begin{dispExample*}{sidebyside}
		\ascboxJ[F]{This is my title}
		
		\ascboxJ*[F]{This is my title}
	\end{dispExample*}
	
	\begin{dispExample*}{sidebyside}
		\ascboxJ[G]{This is my title}
		
		\ascboxJ*[G]{This is my title}
	\end{dispExample*}
	
	\begin{dispExample*}{sidebyside}
		\ascboxJ[H]{This is my title}
		
		\ascboxJ*[H]{This is my title}
	\end{dispExample*}
	
	
	
	\begin{dispListing*}{colback=listcolor,colframe=listred}
		\ascboxK<star>[<type>][<options>]<star>{<title>}
	\end{dispListing*}
	
	\begin{dispExample*}{sidebyside}
		\ascboxK{This is my title}
		
		\ascboxK*{This is my title}
	\end{dispExample*}
	
	\begin{dispExample*}{sidebyside}
		\ascboxK[B]{This is my title}
		
		\ascboxK*[B]{This is my title}
	\end{dispExample*}
	
	
	
	\begin{dispExample*}{sidebyside}
		\ascboxK[C]{This is my title}
		
		\ascboxK*[C]{This is my title}
	\end{dispExample*}
	
	
	
	\begin{dispExample*}{sidebyside}
		\ascboxK[D]{This is my title}
		
		\ascboxK*[D]{This is my title}
	\end{dispExample*}
	
	
	\begin{dispExample*}{sidebyside}
		\ascboxK[E]{This is my title}
		
		\ascboxK*[E]{This is my title}
	\end{dispExample*}
	
	\begin{dispExample*}{sidebyside}
		\ascboxK[F]{This is my title}
		
		\ascboxK*[F]{This is my title}
	\end{dispExample*}
	
	\begin{dispExample*}{sidebyside}
		\ascboxK[G]{This is my title}
		
		\ascboxK*[G]{This is my title}
	\end{dispExample*}
	
	\begin{dispExample*}{sidebyside}
		\ascboxK[H]{This is my title}
		
		\ascboxK*[H]{This is my title}
	\end{dispExample*}
	
	
	\begin{dispListing*}{colback=listcolor,colframe=listred}
		\ascboxL<star>[<type>][<options>]<star>{<title>}
	\end{dispListing*}
	
	\begin{dispExample*}{sidebyside}
		\ascboxL{This is my title}
		
		\ascboxL*{This is my title}
	\end{dispExample*}
	
	
	\begin{dispExample*}{sidebyside}
		\ascboxL[B]{This is my title}
		
		\ascboxL*[B]{This is my title}
	\end{dispExample*}
	
	
	
	\begin{dispExample*}{sidebyside}
		\ascboxL[C]{This is my title}
		
		\ascboxL*[C]{This is my title}
	\end{dispExample*}
	
	
	
	\begin{dispExample*}{sidebyside}
		\ascboxL[D]{This is my title}
		
		\ascboxL*[D]{This is my title}
	\end{dispExample*}
	
	
	\begin{dispExample*}{sidebyside}
		\ascboxL[E]{This is my title}
		
		\ascboxL*[E]{This is my title}
	\end{dispExample*}
	
	
	\begin{dispExample*}{sidebyside}
		\ascboxY{This is my title}
	\end{dispExample*}
	
	
\end{document} 