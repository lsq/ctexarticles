% !TeX encoding = utf-8
% !TEX spellcheck = en-us
% !TeX program = xelatex

\documentclass[]{ctexart}

\usepackage{pst-fractal}

\title{\texttt{pst-fractal}绘制常见分形图的宏包}
\begin{document}
	
	\maketitle
	\centering
	
	\section{康托集 (Cantor set)}
	
	\begin{pspicture}(10,-2)
		\psCantor[linewidth=3mm,linecolor=red,
		n=7,xWidth=11,yWidth=4mm]
	\end{pspicture}
	
	\section{三角形}
	
	\multido{\iA=1+1}{6}{%
		\begin{pspicture}(2,1.7)
			\psSier[linecolor=blue!70,
			fillcolor=red!40](0,0){2cm}{\iA}
	\end{pspicture} }
	
	\section{Siepinski曲线}
	
	\begin{pspicture}(-4,-4)(4,4)
		\psframe*[linecolor=-yellow](-4,-4)(4,4)
		\psSier[n=5,unit=0.125,fillstyle=solid,fillcolor=-cyan,linecolor=-blue]
	\end{pspicture}
	
	\section{Julia集}
	
	
	\psfractal[xWidth=4cm,yWidth=4cm, baseColor=white
	, dIter=20](-2,-2)(2,2)
	
	\section{Mandelbrot集}
	
	
	\psfractal[type=Mandel, xWidth=6cm,
	yWidth=4.8cm, baseColor=white,
	dIter=10](-2,-1.2)(1,1.2)
	
	
	\section{叶序 (Phyllotaxis)}
	
	\psframebox{%
		\begin{pspicture}(-3,-3)(3,3)
			\psPhyllotaxis
	\end{pspicture}}
	
	
	\section{Fern}
	
	\psframebox{%
		\begin{pspicture}(-3,0)(3,11)
			\psFern[scale=30,maxIter=100000,linecolor=green]
	\end{pspicture}}
	
	
	% \section{阿波罗尼圆}
	
	% \begin{pspicture}(-5,-5)(5,5)
	% \psAppolonius[Radius=5cm,Color]
	% \end{pspicture}
	
	\section{毕达哥拉斯树}
	
	\begin{pspicture}[showgrid=true](-6,0)(6,7)
		\psPTree[xWidth=1.75cm,Color=true]
		\psdot*[linecolor=white](0,0)
	\end{pspicture}
	
	\section{斐波那契分形}
	
	
	\begin{pspicture}[showgrid](0,0)(12,8)
		\psFibonacci[unit=0.02,linecolor={[rgb]{0.5 0 0}},n=23,linewidth=0.015cm]
		\rput(5.5,4){n=23}
	\end{pspicture}
	
	\bigskip
	
	\begin{pspicture}(-4,-4)(4,4)
		\psgrid[gridlabels=0pt,subgriddiv=0,gridcolor={[rgb]{0 0 0.5}},griddots=10]
		\pskFibonacci[unit=0.2,linecolor={[rgb]{0 0 0.5}},n=4,k=6,angle=60](-2,0)
	\end{pspicture}
	\bigskip
	
	\begin{pspicture}(-4,-4)(4,4)
		\psgrid[gridlabels=0pt,subgriddiv=0,gridcolor=red,griddots=10]
		\pskFibonacci[unit=0.025,linecolor={[rgb]{0 0 0.5}},linewidth=0.02cm,n=6,k=6](3,0.5)
	\end{pspicture}
	
	\bigskip
	
	\begin{pspicture}[showgrid=false](-4,-4)(4,4)
		\psgrid[gridlabels=0pt,subgriddiv=0,gridcolor=red,griddots=10]
		\psBiperiodicFibonacci[unit=0.2,linecolor={[rgb]{0 0.5 0}},linewidth=0.1cm,n=5,a=6,b=6,angle
		=60](0,2.1)
		\psBiperiodicFibonacci[unit=0.2,linecolor=white,n=5,a=6,b=6,angle=60](0,2.1)
	\end{pspicture}
	
	\bigskip
	
	\begin{pspicture}(-5,-5)(5,5)
		\psBiperiodicFibonacci[unit=0.8,linecolor=black,linewidth=0.1cm,,n=8,a=2,b=3,angle=120](-1,1)
		\psBiperiodicFibonacci[unit=0.8,linecolor=white,n=8,a=2,b=3,angle=120](-1,1)
	\end{pspicture}
	
	
	\section{The Henon Sttractor}
	\begin{pspicture}(-5,-6)(5,6)
		\psclip{\psframe(-5,-5)(5,5)}
		\psHenon
		\endpsclip
		\psgrid[unit=5,subgriddiv=10](-1,-1)(1,1)
	\end{pspicture}
	
\end{document}