\documentclass[zihao=-4]{ctexart}
\usepackage{graphicx}
\usepackage{fancyhdr}
\usepackage{lastpage}
\usepackage[dvipsnames]{xcolor}
\usepackage{tikz}
\usetikzlibrary{calc}
\usepackage{anyfontsize}
\usepackage{sectsty}
\usepackage{colortbl}
\usepackage{hhline}
\usepackage{newtxtext}
\usepackage{hyperref}
\usepackage{longtable,booktabs}
\usepackage{pdfpages}


\definecolor{mgray}{rgb}{0.498,0.498,0.498}
%\usepackage[left=19.1mm,right=19.1mm,top=25.4mm,bottom=25.4mm]{geometry}% 用于页面设置
% 设置为A4纸,边距适中模式(参考永中office)
%\geometry{
%	width = 210mm,
%	height = 297mm,
%	left = 19.1mm,
%	right = 19.1mm,
%	top = 25.4mm,
%	bottom = 25.4mm
%}
%————————————————
%版权声明:本文为CSDN博主「夜间之路」的原创文章,遵循CC 4.0 BY-SA版权协议,转载请附上原文出处链接及本声明。
%原文链接:https://blog.csdn.net/con_me/article/details/114444542
\usepackage[left=19.1mm, right=19.1mm, top=25.4mm, bottom=25.4mm]{geometry}
\pagestyle{fancy}
\lhead{\raisebox{1mm}{\includegraphics[scale=0.4]{./maxeye.png}\zihao{5}}} %在此处插入logo.pdf图片 图片靠左
%\chead{MAXEYE} % 页眉中间位置内容
\rhead{\bfseries 文档控制} %页眉右边位置内容,并加粗
\lfoot{\zihao{-5}\copyright 2020-2021深圳市千分一智能技术有限公司\ 版权所有\  未经授权\ 禁止使用}  %页脚
\cfoot{}
\rfoot{\zihao{-5}第\thepage,共\pageref{LastPage}页}
\renewcommand{\headrulewidth}{2pt}%脚注线的宽度
\renewcommand{\footrulewidth}{1pt}%脚注线的宽度
%\setlength{\abovecaptionskip}{-1cm} 

\title{\hypertarget{ux5173ux4e8eux4f9bux5e94ux5546ux4ed8ux6b3eux7533ux8bf7ux5355ux4ed8ux6b3eux4fe1ux606fux586b}{%
		{\heiti 付款申请单付款信息填写说明}\label{ux5173ux4e8eux4f9bux5e94ux5546ux4ed8ux6b3eux7533ux8bf7ux5355ux4ed8ux6b3eux4fe1ux606fux586b}}}
%\setCJKmainfont[AutoFakeBold]{FangSong_GB2312}
\setCJKmainfont[AutoFakeBold]{FangSong_GB2312}
\author{}
\date{}
\begin{document}
%	\pagestyle{empty}
%	
%	\begin{tikzpicture}[overlay,remember picture]
%		
%		% Background color
%		\fill[
%		black!2]
%		(current page.south west) rectangle (current page.north east);
%		
%		% Rectangles
%		\shade[
%		left color=Dandelion,
%		right color=Dandelion!40,
%		transform canvas ={rotate around ={45:($(current page.north west)+(0,-6)$)}}]
%		($(current page.north west)+(0,-6)$) rectangle ++(9,1.5);
%		
%		\shade[
%		left color=lightgray,
%		right color=lightgray!50,
%		rounded corners=0.75cm,
%		transform canvas ={rotate around ={45:($(current page.north west)+(.5,-10)$)}}]
%		($(current page.north west)+(0.5,-10)$) rectangle ++(15,1.5);
%		
%		\shade[
%		left color=lightgray,
%		rounded corners=0.3cm,
%		transform canvas ={rotate around ={45:($(current page.north west)+(.5,-10)$)}}] ($(current page.north west)+(1.5,-9.55)$) rectangle ++(7,.6);
%		
%		\shade[
%		left color=orange!80,
%		right color=orange!60,
%		rounded corners=0.4cm,
%		transform canvas ={rotate around ={45:($(current page.north)+(-1.5,-3)$)}}]
%		($(current page.north)+(-1.5,-3)$) rectangle ++(9,0.8);
%		
%		\shade[
%		left color=red!80,
%		right color=red!80,
%		rounded corners=0.9cm,
%		transform canvas ={rotate around ={45:($(current page.north)+(-3,-8)$)}}] ($(current page.north)+(-3,-8)$) rectangle ++(15,1.8);
%		
%		\shade[
%		left color=orange,
%		right color=Dandelion,
%		rounded corners=0.9cm,
%		transform canvas ={rotate around ={45:($(current page.north west)+(4,-15.5)$)}}]
%		($(current page.north west)+(4,-15.5)$) rectangle ++(30,1.8);
%		
%		\shade[
%		left color=RoyalBlue,
%		right color=Emerald,
%		rounded corners=0.75cm,
%		transform canvas ={rotate around ={45:($(current page.north west)+(13,-10)$)}}]
%		($(current page.north west)+(13,-10)$) rectangle ++(15,1.5);
%		
%		\shade[
%		left color=lightgray,
%		rounded corners=0.3cm,
%		transform canvas ={rotate around ={45:($(current page.north west)+(18,-8)$)}}]
%		($(current page.north west)+(18,-8)$) rectangle ++(15,0.6);
%		
%		\shade[
%		left color=lightgray,
%		rounded corners=0.4cm,
%		transform canvas ={rotate around ={45:($(current page.north west)+(19,-5.65)$)}}]
%		($(current page.north west)+(19,-5.65)$) rectangle ++(15,0.8);
%		
%		\shade[
%		left color=OrangeRed,
%		right color=red!80,
%		rounded corners=0.6cm,
%		transform canvas ={rotate around ={45:($(current page.north west)+(20,-9)$)}}]
%		($(current page.north west)+(20,-9)$) rectangle ++(14,1.2);
%		
%		% Year
%		%		\draw[ultra thick,gray]
%		%		($(current page.center)+(5,2)$) -- ++(0,-3cm)
%		%		node[
%		%		midway,
%		%		left=0.25cm,
%		%		text width=5cm,
%		%		align=right,
%		%		black!75
%		%		]
%		%		{
%		%			{\fontsize{25}{30} \selectfont \bf ANNUAL \\[10pt] REPORT}
%		%		}
%		%		node[
%		%		midway,
%		%		right=0.25cm,
%		%		text width=6cm,
%		%		align=left,
%		%		orange]
%		%		{
%		%			{\fontsize{72}{86.4} \selectfont 2020}
%		%		};
%		
%		% Title
%		\node[align=center] at ($(current page.center)+(0,-5)$)
%		{
%			{\fontsize{40}{48} \selectfont {{\heiti 多组织产能协同委托加工流程}}} \\[1cm]
%			{\fontsize{16}{19.2} \selectfont \textcolor{orange}{ \heiti 深圳市千分一智能技术有限公司}}\\[3pt]
%			%			Company Name\\[3pt]
%			%			Address
%		};
%	\end{tikzpicture}
	
	%	\clearpage
	\newpage
	\pagestyle{fancy}% 设置页眉
	{\noindent \zihao{3}\bf 文档控制\\
	\zihao{-3} 更改记录
	\zihao{-4}
	\setlength\doublerulesep{2pt}\doublerulesepcolor{mgray}
	
	\begin{tabular}{|p{2cm}|p{3cm}|p{2cm}|p{8cm}|}
		\hline
		\rowcolor[rgb]{0.898,0.898,0.898} \textbf{日期} & \textbf{作者} & \textbf{版本} & \textbf{更改参考}\\\hhline{|=|=|=|=|}
		& & & \\\hline
		& & & \\\hline
		& & & \\\hline
	\end{tabular}\\
	\zihao{-3} 审核
	\zihao{-4}
	
	\begin{tabular}{|p{2cm}|p{3cm}|p{10.5cm}|}
		\hline
		\rowcolor[rgb]{0.898,0.898,0.898} \textbf{姓名} & \textbf{职位} & \textbf{签字} \\\hhline{|=|=|=|}
		& & \\\hline
		& & \\\hline
		& & \\\hline
	\end{tabular}\\	
	\zihao{-3} 分发
	\zihao{-4}
	
	\begin{tabular}{|p{2cm}|p{3cm}|p{10.5cm}|}
		\hline
		\rowcolor[rgb]{0.898,0.898,0.898} \textbf{拷贝号} & \textbf{姓名} & \textbf{区域} \\\hhline{|=|=|=|}
		& & \\\hline
		& & \\\hline
		& & \\\hline
	\end{tabular}
}
	\newpage
	%\clearpage
%	\thispagestyle{empty}
	\rhead{\leftmark}
	\tableofcontents
	\newpage
	
	\zihao{-4}
%	%	\pagestyle{fancy}% 设置页眉
%	\section{介绍}
%	
%	\section{操作流程}
%	
%	\subsection{单据类型设置}
%	设置单据类型中生产订单单据中“名称”为“组织委托加工-汇报入库”,”参数设置“中”单据控制“勾选”组织间受托加工“。
%	
%	%	\subsection{设置BOM}
%	设置BOM中“物料控制”页签中“发料组织”为“深圳市千分一智能技术有限公司”,及”默认发料仓库“,”货主“为“深圳市千分一智能技术有限公司”,”供料方式“为”委托方供料“。
	
	%	\section{附件}



\maketitle
\hypertarget{ux586bux5199ux8bf4ux660e-1}{%
\section{填写举例}\label{ux586bux5199ux8bf4ux660e-1}}

以外币付款说明为例,人民币相对简单点,方法类似,下面是一个外币的Payment Instruction。\\
\begin{figure}[!h]
	\centering
%	\includegraphics[width=\ScaleIfNeeded]{./payments_instruction.pdf}
%	\rotatebox{10}{\includegraphics[scale=0.5]{./payments_instruction.pdf}}
	\includegraphics[height=0.5\paperheight, angle=0.2]{./payments_instruction.pdf}
%	\caption{Schematical view of Spearman's theory.}
\end{figure}\\
系统中供应商资料中“财务信息”填写结果如下:\\
\begin{figure}[!h]
	\centering
	%	\includegraphics[width=\ScaleIfNeeded]{./payments_instruction.pdf}
	%	\rotatebox{10}{\includegraphics[scale=0.5]{./payments_instruction.pdf}}
	\includegraphics[width=\linewidth]{./fksq_usd.png}
	%	\caption{Schematical view of Spearman's theory.}
\end{figure}\\
\hypertarget{ux4f9bux5e94ux5546ux57faux7840ux8d44ux6599}{%
\section{填写供应商基础资料}\label{ux4f9bux5e94ux5546ux57faux7840ux8d44ux6599}}
以下是上图中的供应商资料中字段与付款信息对应关系说明\\
\hypertarget{ux586bux5199ux8bf4ux660e-2}{%
\subsection{字段说明:}\label{ux586bux5199ux8bf4ux660e-2}}

\begin{longtable}[]{@{}ll@{}}	
\toprule
\textbf{银行信息} & \textbf{填写对应列名称} \\
\midrule
\endhead
Beneficairy's Bank Name & 开户银行 \\
Beneficairy's Bank Address & 开户行地址 \\
Benefcairy's Name & 账户名称 \\
Benefcairy's Address & 地址(表头) \\
Account Number(USD) & 银行账号 \\
Swift Code & SwiftCode \\	
ABA(ACH) CODE / Beneficairy's Bank Code & 联行号 \\
Bank Country & 国家 \\
\bottomrule
\end{longtable}

\hypertarget{ux5176ux4ed6ux6ce8ux610fux5730ux65b9}{%
\subsection{其他注意地方}\label{ux5176ux4ed6ux6ce8ux610fux5730ux65b9}}

\begin{enumerate}
\def\labelenumi{\arabic{enumi}.}
\item
  如果是\textbf{外币},\textbf{结算币别}填写:“美元”/“日元”等。
\item
  \textbf{结算方式}:“电汇”。
\item
  \textbf{付款条件},根据付款账款选择。
\item
  \textbf{发票类型}也要根据实际情况选择,进口请选择\textbf{零税率}。
\item
  如果有\textbf{多个}银行资料信息,可以新增行,添加另外一个银行账号资料,但需要选择一个作为\textbf{默认}。
\end{enumerate}

\hypertarget{ux5957ux6253ux8bbeux7f6eux9009ux62e9}{%
\section{套打设置---自动选择套打模板}\label{ux5957ux6253ux8bbeux7f6eux9009ux62e9}}
如果想在打印付款申请单时,根据币种自动选择打印模板,需要如下设置:\\
\textbf{前置条件}:在付款申请单先选择"\textbf{币别}",只有选择正确,套打设置才能\textbf{自动}选择正确的\textbf{套打模板}。\\
%\begin{figure}[!h]
%	\centering
	%	\includegraphics[width=\ScaleIfNeeded]{./payments_instruction.pdf}
	%	\rotatebox{10}{\includegraphics[scale=0.5]{./payments_instruction.pdf}}
	\includegraphics[width=\linewidth]{./fksq_print0.png}
	%	\caption{Schematical view of Spearman's theory.}
%\end{figure}

\begin{enumerate}
\def\labelenumi{\arabic{enumi}.}
\item
  在付款申请单菜单栏,“选项”-\/-\/-“套打设置”中清空\textbf{默认}套打设置。\\
%  \begin{figure}[!h]
%  	\centering
  	%	\includegraphics[width=\ScaleIfNeeded]{./payments_instruction.pdf}
  	%	\rotatebox{10}{\includegraphics[scale=0.5]{./payments_instruction.pdf}}
  	\includegraphics[width=\linewidth]{./fksq_print.png}
  	%	\caption{Schematical view of Spearman's theory.}
%  \end{figure}\\
\item
  在“高级”中新增行,并按如下操作设置套打模板。\\
%  \begin{figure}[!h]
%  	\centering
  	%	\includegraphics[width=\ScaleIfNeeded]{./payments_instruction.pdf}
  	%	\rotatebox{10}{\includegraphics[scale=0.5]{./payments_instruction.pdf}}
  	\includegraphics[height=0.5\textheight,width=\linewidth]{./fksq_print2.png}\\
  	%	\caption{Schematical view of Spearman's theory.}
%  \end{figure}\\
%  \begin{figure}[!h]
%  	\centering
  	%	\includegraphics[width=\ScaleIfNeeded]{./payments_instruction.pdf}
  	%	\rotatebox{10}{\includegraphics[scale=0.5]{./payments_instruction.pdf}}
  	\includegraphics[width=\linewidth]{./fksq_print1.png}
  	%	\caption{Schematical view of Spearman's theory.}
%  \end{figure}\\
\end{enumerate}

\end{document}
