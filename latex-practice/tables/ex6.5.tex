\documentclass{ctexart}
\usepackage{color}
\usepackage{dashbox}

\begin{document}

This example 6.5

\begin{tabular}{|l|c|r|}
	\hline
	\multicolumn{3}{|c|}{Sample Tablular}\\\hline
	col head & col head & col head \\\hline
	Left & centered & right\\\cline{1-2}
	aligned & items & aligned\\\cline{2-3}
	items & items & items\\\cline{1-2}
	left item & centered & right\\\hline
\end{tabular}

example 6.6

\begin{tabular}{c r @{.} l}
	\hline
	太阳系中的行星 & \multicolumn{2}{c}{赤道半径km}\\\hline
	水星 & 2 & 44\\
	金星 & 6 & 1\\
	地球 & 6\:378 & 142\\\hline
\end{tabular}

example 6.7

\newcommand{\bb}[1]{\raisebox{-7ex}[0pt][0pt]{\shortstack{#1}}}
%\bb{原子\\序数}

\begin{tabular}{|c|c|c|c|}
	\hline
	\multicolumn{4}{|c|}{\parbox[c][9mm]{0pt}{}%
	\kaishu\bfseries 前5种化学元素的电子状态分布}\\\hline
	\bb{原子\\序数}   & \bb{元素\\符号} & \multicolumn{2}{c|}{各电子层上的电子数}\\\cline{3-4}
	& & K层,$n=1$ 	& L层,$n=2$ \\\cline{3-4}
	& & $s$ 亚层	 	& $s$ 亚层 $p$亚层\\\cline{3-4}
	& & $l=0$        & $l=0$\quad $l=1$ \\\hline
	1 & H 			 & 1 & \\
	2 & He 			 & 2 & \\\cline{2-3}
	3 & Li		     & 2 & 1\qquad ~\\
	4 & Be	         & 2 & 2\qquad ~\\
	5 & B		     & 2 & 2\qquad 1\\\hline
	
\end{tabular}

For emphasis, you may wish to \raisebox{2.5ex}[12pt]{raise} or \raisebox{-1.0ex}{lower} certain text inside your documents.

\noindent \makebox[0pt][r]{\fbox{注意}}\qquad 如果重新安装操作系统,则系统盘中所有数据将被删除!

\newlength{\Mylen}
\settowidth{\Mylen}{勾股定理}
勾股定理:
\makebox[0pt][l]{	%
\color{blue}\rule[-0.9ex]{\Mylen}{1pt}}
直角三角形两直角边的平方和等于斜边的平方。

\fbox{系统}\dbox{系统}\dashbox[80mm][c]{系统}
\end{document}